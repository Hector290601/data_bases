\newpage
\part{\acrlong{secondSection}}
\section{Definici\'{o}n}
Son datos, que contextualmente deben tener relaci\'{o}n, f\'{i}sicamente pueden o no tener relaci\'{o}n y se almacenan en alg\'{u}n lugar.\\
Son conexiones de datos relacionados.\\
Pueden tener relaci\'{o}n f\'{i}sica o contextual.\\
La informaci\'{o}n por s\'{i} sola es una cosa, pero el SGDB (Sistema gestor de bases de datos).\\
\textit{El SGDB}

\section{Arquitectura BBDD Oracle}
\begin{itemize}
	\item Multitenant Architecture.
	\item Services and connections.
	\item Data Dictionary Architecture.
	\item Security in CDB.
	\item Managing a CDB.
\end{itemize}
Tenant = Inquilino.\\
Con \'{e}ste esquema, el servidor se pude utilizar de mejor forma, evitando futuros problemas de migraciones.\\
Es como vivir en un edificio de departamentos, permite que exista mejor comunicaci\'{o}n entre DDBB.\\
En la estructura \textit{tenant} se puede simplemente utilizar las instalaciones ya hechas, sin necesidad de reinstalar todo.\\

\section{Hoja de ruta de versiones DDBB Oracle}
\begin{center}
	\begin{tabular}{| c | c | c | c | c | c | c |}
		\hline
		2009 & 11.2.0.4 LT& & & & & \\
		\hline
		2010 & LT & & & & & \\
		\hline
		2011 & LT & & & & & \\
		\hline
		2012 & LT& & & & & \\
		\hline
		2013 & LT & & & & & \\
		\hline
		2014 & LT & & & & & \\
		\hline
		2015 & CEPS & & & & & \\
		\hline
		2016 & CEPS & & & & & \\
		\hline
		2017 & CEPS& & & & & \\
		\hline
		2018 & CEPS & & & & & \\
		\hline
		2019 & CESE & & & & & \\
		\hline
		2020 & CESE & & & & & \\
		\hline
		2021 & CESM & & & & & \\
		\hline
		2022 & CESM & & & & & \\
		\hline
		2023 & & & & & & \\
		\hline
		2024 & & & & & & \\
		\hline
		2025 & & & & & & \\
		\hline
		2026 & & & & & & \\
		\hline
		2027 & & & & & & \\
		\hline
	\end{tabular}
\end{center}
Las versione de Oracle se nombran a partir del a\~{n}o en el que salen.

\section{PDBs}
Pluggable Databases (PDBs)
Tienen tres capas:
\begin{itemize}
	\item CDB (Multi containter database) \textit{ROOT}
	\item PDB (seed) \textit{PDB\$SEED}
	\item  PDBs PDB applications as if it were a non-CBD.
\end{itemize}

\textbf{Every CDB contains exactly one root, exactly one seed PDB and zero or more user-created PDBs}

Las arquitecturas multitenant pueden albergar hasta 252 inquilinos te\'{o}ricos.\\
El n\'{u}mero de inquilinos reales se obtiene de repartir los recursos del servidor a las DDBB.\\
La arquitectura EXADATA nos permite tener servidores con mejores y mayores prestaciones.\\
La arquitectura \textit{tenant} puede subirse a la nube, con Oracle Cloud, Azure, AWS, GCS, SAP, etc.\\

\section{Arquitectura de un contenedor de bases de datos}
\subsection{Componentes principales}
\begin{itemize}
	\item Procesos
	\item Memoria
	\item Datos
\end{itemize}
\subsection{Single DB shares}
\begin{itemize}
	\item BG process.
	\item Shared/process memory.
	\item Oracle metadata.
	\item Redo Log files.
	\item Control files.
	\item Some datafiles.
\end{itemize}
\subsection{CDA}
\begin{itemize}
	\item Server
	\begin{itemize}
		\item Instance
			\begin{itemize}
				\item Sustem Gloab Area
				\begin{itemize}
					\item PDBd2
					\item PDBd3
					\item PDBd4
					\item PDBd2
				\end{itemize}
				\item Process structures
			\end{itemize}
		\item Container database
			\begin{itemize}
				\item Root Container
				\begin{itemize}
					\item Datafiles
					\begin{itemize}
						\item System
						\item SysAux
					\end{itemize}
					\item Undo
					\item Temp
					\item Control files
					\item Redo Log Files
				\end{itemize}
			\end{itemize}
		\item PDBs
	\end{itemize}
\end{itemize}
Se genera un espacio en disco que genera un archivo de UNDO\\
Los respaldos deben ser probados para respaldarlos correctamente.\\
Para recuperar la informaci\'{o}n se debe buscar un respaldo previo o completo antes de la fecha deseada y posteriormente aplicar los \textit{Redo Logs}\\
Las consultas no se almacenan en los \textit{redo logs}.\\
Las DDBB \textit{in memory} pueden ser peligrosas debido a que son muy demandantes de la RAM.\\
Su factor de compresi\'{o}n m\'{a}ximo es de 1.7x\\

