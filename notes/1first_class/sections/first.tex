\newpage
\part{\acrlong{firstSection}}
\chapter{Datos generales}

\section{Sobre el profesor}
\begin{itemize}
	\item Ingeniero en sistemas computacionales.
	\item ESCOM del IPN.
	\item Maestro en administraci\'{o}n de tecnolog\'{i}as de la informaci\'{o}n por el ITESM.
	\item PMP certificado desde el 2013 y revalidado en el 2021.
	\item Oracle DB Certified by Oracle.
\end{itemize}
\\
Clases Viernes y S\'{a}bados\\.
\section{Objetivo general}
\begin{itemize}
	\item Conocer y manejar diferentes modelos de DDBB.
	\item Manejar la arquitecrua de una DB y su modelo de consulta.
	\item Ser capaz de construir aplicaciones en el mundo real que manejen DDBB.
	\item Conocer la importancia del ingl\'{e}s en relaci\'{o}n a la documentaci\'{o}n t\'{e}cnica.
\end{itemize}

\section{Reglas}
\begin{itemize}
	\item Prohibido copiar.
	\item El examen final no se puede presentar para mejorar la calificaci\'{o}n.
	\item La materia se excenta con al menos 6.
	\item Necesario acreditar el laboratorio para acreditar la teor\'{i}a.
	\item Est\'{a} prohibido el uso del celular para redes sociales.
\end{itemize}

\section{Temario}
\begin{enumerate}
	\item Introducci\'{o}n a las DDBB.
		\begin{itemize}
			\item Bases de datos y SGDB.
			\item Arquitectura de una SGDB.
			\item DB control langage.
			\item Tipos de DDBB.
			\item Integridad, redundancia e inconsistencia de datos.
			\item Modelos de bases de datos (Jer\'{a}rquicos, Red, Relaciona, Entidad-relaci\'{o}n, OODBMS, ORDBMS, No Only SQL).
			\item Metodolog\'{i}as empleadas en el dise\~{n}o de DDBB.
		\end{itemize}
	\item Dise\~{n}o conceptual de una DB.
		\begin{itemize}
			\item Definici\'{o}n de Modelo E/R.
			\item Representaci\'{o}n de entidades y atributos.
			\item Representaci\'{o}n de relaciones.
		\end{itemize}
	\item Modelo relacional.
	\item Dise\~{n}o l\'{o}gico de una DB.
		\textbf{Primer examen parcial, 17 de Marzo}
	\item Normalizaci\'{o}n.
	\item Dise\~{n}o f\'{i}sico de una DB.
	\item Lenguaje de consulta de datos (DQL).
		\textbf{Segundo examen parcial, 5 de mayo}
	\item Introducci\'{o}n a la programaci\'{o}n en bases de datos.
		\textbf{Tercer examen parcial 27 de mayo.}
\end{enumerate}

\section{Proyecto final}
El proyecto se presenta el 2 y 3 de junio.\\
\begin{itemize}
	\item Hacer una propuesa ejecutiva (a elegir).
	\item Diagrama de modelo relacional.
	\item Arvhivos fuente de proyecto y/o scripts.
	\item Resultado y/o prototipo final del proyeccto.
	\item Presentaci\'{o}n del proyecto (calificaci\'{o}n individual)
\end{itemize}
\textbf{Buscar qu\'{e} son los Oracle Labs}

\section{Tareas}
\begin{itemize}
	\item Las tareas se elaboran en formato electr\'{o}nico.
	\item Deben entregarse mediante MS Teams dentro del plazo establecido.
	\item Son individuales (excepto si se indica lo contrario).
	\item Los protectos deben entregarse mediante MS Teams dentro del plazo establecido.
	\item R\'{u}brica de evaluaci\'{o}n de tareas:
		\begin{itemize}
			\item Protada.
			\item Introducci\'{o}n.
			\item Desarrollo.
			\item Conclusiones.
			\item Bibliograf\'{i}a en formato APA.
		\end{itemize}
\end{itemize}

\section{Forma de evaluaci\'{o}n}
\begin{center}
	\begin{tabular}{| c | c | c |}
		\hline
		Ex\'{a}menes parciales & 50 & Haber presentado los tres ex\'{a}menes\\
		& & y 80 de tareas.\\
		\hline
		Pr\'{a}cticas de labpratorio & 20 & pasar el laboratorio\\
		\hline
		Tareas de investigaci\'{o}n & 10 & \\
		\hline
		Proyecto Final & 20 & Presentar el 100 del proyecto\\
		\hline
	\end{table}
\end{center}
\textbf{Se excenta con al menos 65\%}

Asistencia obligatoria de al menos 75\% de asistencia.
La medici\'{o}n del aprendizaje se hace de acuerdo a las participaciones.
Contacto:
francisto.rojas"@"fi.unam.edu
